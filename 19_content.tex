% =========================================================================== %

\begin{frame}[t,plain]
\titlepage
\end{frame}

% =========================================================================== %

\begin{frame}{Recap}
%
\begin{columns}[T]
\column{.45\linewidth}
\textbf{MatPlotLib}
\begin{itemize}
\item Commands \texttt{plt.plot(X, Y)} and \texttt{plt.show()}
\item \texttt{plt.bar}, \texttt{plt.scatter}, \texttt{plt.quiver}, ...
\item Millions of optional arguments
\item Plot-Objekts \texttt{figure} and \texttt{axis}
\item In doubt: Look it up on \url{https://matplotlib.org/} or in the lecture notes
\end{itemize}
%
\column{.45\linewidth}
\textbf{NumPy}
\begin{itemize}
\item Central element: \texttt{np.ndarray}
\item Componentwise operations
\item Create from Python \texttt{list}s or using dedicated functions
\item Proper math library for fast vectorized operations
\item Proper commands for reduction (\eg \texttt{np.sum})
\end{itemize}

\end{columns}
%
\begin{center}
	\emph{Any Questions?}
\end{center}
%
\end{frame}

% =========================================================================== %

\begin{frame}[fragile]
%
\begin{codebox}[Exercise: Linear Optimization]
\begin{minted}[fontsize=\scriptsize, linenos]{python}
# the givens
money   = 10.0

names   = \
    ["dark chocolate", "milk chocolate", "caramel nuts bar", "coco cubes", "mint drops"] 

prices  = np.array(
    [            1.0 ,             0.9 ,               2.3 ,         1.5 ,          0.3]
)

weights = np.array(
    [              5 ,               4 ,                12 ,           8 ,            1]
)

N       = len(names)
\end{minted}
\end{codebox}
%
\end{frame}

% =========================================================================== %

\begin{frame}[fragile]
%
\begin{tcbraster}[raster columns=2,
                  raster equal height,
                  nobeforeafter,
                  raster column skip=0.2cm]
\begin{codebox}[Exercise: Linear Optimization]
\begin{minted}[fontsize=\scriptsize, linenos, firstnumber=last]{python3}
# construction of the meshgrid
upperLimits = money // prices

values = tuple(
    [np.arange(u + 1) 
     for u in upperLimits]
)

meshgrid = np.meshgrid(*values)

# rempark: 
# *tuple
# automatically unpacks a tuple. Hence,
# np.meshgrid(*values)
# is the same as
# np.meshgrid(values[0], values[1], ...)
\end{minted}
\end{codebox}
%
\begin{tcolorbox}[title=Elements in Code]
\begin{itemize}
\item \texttt{upperLimits}: this many elements can be bought if we limit ourselves to a single sort of sweets
\item \texttt{values}: \inPy{tuple} of lists: each lists goes from 0 to \texttt{N\_sweet} ...
\item ... which can be used to build the mesgrid from this.
\item \texttt{meshgrid}: \inPy{tuple} of tensors
\end{itemize}
\end{tcolorbox}
\end{tcbraster}
%
\end{frame}

% =========================================================================== %

\begin{frame}{Visualizing Meshgrids}
%
\begin{itemize}
\item Understand meshgrid as: \emph{all conceivable combinations of the numbers in the lists}
\item One tensor per list
\item Each tensor holds the numbers for \emph{one} component (\ie sweet)
\item \texttt{meshgrid[0]} is the tensor with the $0^{\text{th}}$ component (\ie dark chocolate)
\end{itemize}
%
\end{frame}

% =========================================================================== %

\begin{frame}[fragile]
%
\begin{tcbraster}[raster columns=2,
                  raster equal height,
                  nobeforeafter,
                  raster column skip=0.2cm]
\begin{codebox}[Example: 2D Meshgrids]
\begin{minted}[fontsize=\scriptsize, linenos]{python3}
import numpy as np

countDark = np.array(range(5))
    # [0 1 2 3 4]
countMilk = np.array(range(3))
    # [0 1 2]
    
meshgrid = np.meshgrid(
    countDark, countMilk
)

print("dark chocolate")
print(meshgrid[0])
print()

print("milk chocolate")
print(meshgrid[1])
\end{minted}
\end{codebox}
%
\begin{cmdbox}[Output: 2D Meshgrids]
\begin{minted}[fontsize=\scriptsize]{text}
dark chocolate
[[0 1 2 3 4]
 [0 1 2 3 4]
 [0 1 2 3 4]]

milk chocolate
[[0 0 0 0 0]
 [1 1 1 1 1]
 [2 2 2 2 2]]
\end{minted}
\end{cmdbox}
\end{tcbraster}
%
\end{frame}

% =========================================================================== %

\begin{frame}[fragile]
%
\begin{tcbraster}[raster columns=2,
                  raster equal height,
                  nobeforeafter,
                  raster column skip=0.2cm]
\begin{codebox}[Exercise: Linear Optimization]
\begin{minted}[fontsize=\scriptsize, linenos, firstnumber=32]{python3}
# Find the cost for each config ...
totalMoney = np.zeros(meshgrid[0].shape)
for i, price in enumerate(prices) :
    totalMoney += price * meshgrid[i]

# ... and the point score as well
totalScore = np.zeros(meshgrid[0].shape)
for i, score in enumerate(weights) :
    totalScore += score * meshgrid[i]

# can be done in even fewer lines
# with np.dot ;)
\end{minted}
\end{codebox}
%
\begin{tcolorbox}[title=Elements in Code]
\begin{itemize}
\item \texttt{totalMoney}, \texttt{totalScore}: tensors
\item $N_{\text{dark chocolate}} \times N_{\text{milk chocolate}} \times N_{\text{nut bars}} \times N_{\text{coco cubes}} \times N_{\text{mint drops}}$ components
\item If this blows your mind, picture a simple table with rows (dark chocolate) and columns (milk chocolate)
\end{itemize}
\end{tcolorbox}
\end{tcbraster}
%
\end{frame}

% =========================================================================== %

\begin{frame}[fragile]
%
\begin{codebox}[Exercise: Linear Optimization]
\begin{minted}[fontsize=\scriptsize, linenos, firstnumber=32]{python3}
# which configurations can we actually afford?
mask = totalMoney > money
totalScore[mask] = -1
    # give bad score to configurations that are too expensive

# argmax yields the ID of the best configuration ...
bestID = np.argmax(totalScore)

# ... and unravel_index reconstructs the indices
# in our multidimensional array from this ID
bestConfig = np.unravel_index(bestID, shape=meshgrid[0].shape)
\end{minted}
\end{codebox}
%
\end{frame}

% =========================================================================== %

\begin{frame}[fragile]{Flattened and Multidimensional Indices}
%
\begin{itemize}
\item What we see:\\
	\vspace{3pt}
	$\begin{pmatrix}
		1 & 5 & 4 \\
		2 & 2 & 3
	\end{pmatrix}$
	\vspace{3pt}
\item What NumPy sees:\\
	\texttt{[1 5 4 2 2 3]}
\item Output of \texttt{np.argmax}: Index of maximum in \emph{flattened} array\\
	\texttt{1}
\item Output of \texttt{np.unravel\_index}: \inPy{tuple} with the \enquote{reconstructed} matrix coordinates\\
	\texttt{(0, 1)}
\end{itemize}
%
\end{frame}

% =========================================================================== %

\begin{frame}[fragile]
%
\begin{codebox}[Exercise: Linear Optimization]
\begin{minted}[fontsize=\scriptsize, linenos, firstnumber=last]{python3}
# Output on screen:
print("Sugar-Strategy:")
print("Sweet                | Count  | Price | Score ")
print("---------------------+--------+-------+-------")
for i in range(N) :
    ithComponentCountCount = meshgrid[i][bestConfig]
    print(f"{names[i]:20} |"
          f" {ithComponentCountCount:6.0f} |"
          f" {ithComponentCountCount * prices [i]:5.2f} |"
          f" {ithComponentCountCount * weights[i]:5}"
    )
print("---------------------+--------+-------+-------")
print(f"{'TOTAL':20} |"
      f" {np.sum(bestConfig):6} |"
      f" {totalMoney[bestConfig]:5.2f} |"
      f" {totalScore[bestConfig]:5.0f}"
)
print()
print("Analyzed", np.prod(1 + upperLimits, dtype=np.int), "configurations.")
\end{minted}
\end{codebox}
%
\end{frame}

% =========================================================================== %

\begin{frame}[fragile]{Plan for Today}
%
\begin{itemize}
\item Some Physics: 
	\begin{itemize}
	\item Spin lattice and magnetism
	\item Energy
	\item Boltzmann distribution
	\end{itemize}
\item Representing reality in code
	\begin{itemize}
	\item List of properties and questions
	\item Choosing apt data structures
	\item First methods
	\end{itemize}
\item Metropolis algorithm
	\begin{itemize}
	\item Bit configuration spaces and Monte Carlo methods
	\item Metropolis algorithm \enquote{for humans}
	\item Metropolis algorithm in Python
	\end{itemize}
\end{itemize}
%
\end{frame}

% =========================================================================== %

\begin{frame}{Physics: Spin Lattice}
%
\begin{columns}[T]
\column{.5\linewidth}
\begin{itemize}
\item Picture: atoms as little bar magnets
\item Freely rotatable in space
	\begin{itemize}
	\item Simplification: only two orientations
	\item either: north pole \enquote{up}
	\item or: north pole \enquote{down}
	\end{itemize}
\item Magnetic fields sum up
	\begin{itemize}
	\item Parallel alignments reinforce one another
	\item Antiparalellel alignments cancel out
	\end{itemize}
\item Representation of solid: grid of bar magnets
\end{itemize}
%
\column{.5\linewidth}
\includegraphics[width=.7\linewidth]{./gfx/spinLattice}
\end{columns}
%
\end{frame}

% =========================================================================== %\\

\begin{frame}{Physics: Total Magnetization and Energy in the Lattice}
%
\begin{columns}[T]
\column{.5\linewidth}
\begin{itemize}
\item Total magnetization
	\begin{itemize}
	\item \enquote{How strong is the magnetic field of the entire lattice?}
	\item {}[Number of spin up atoms] minus [number of spin down atoms]
	 \end{itemize} 
\item Total lattice energy
	\begin{itemize}
	\item Magnets attract or repell each other
	\item There is an \enquote{energy cost} to bring equal poles together
	\item Depends on distance
	\item Simplification: Only next neighbours relevant
	\end{itemize}
\end{itemize}
%
\column{.5\linewidth}
\includegraphics[width=.7\linewidth]{./gfx/spinLattice}
\end{columns}
%
\end{frame}

% =========================================================================== %\\

\begin{frame}{Physics: Magnetization Density and Lattice Energy in Equations}
%
\begin{itemize}
\item Represent each magnet by a number $+1$ or $-1$
\item Lattice in its entirety: table (matrix): $G = \left( g_{ij} \right)$
\item Magnetization density
	\[ \small M = \frac{1}{V} \sum_{i, j} g_{ij} \qquad \text{where } V: \text{\enquote{Volume}, number of atoms} \]
	\vspace{-12pt}
\item Energy
	\begin{itemize}
	\item Coupling constant $J$
	\item For each neighbour with \emph{parallel} spin: gain energy $J$
	\item For each neighbour with \emph{antiparallelen} spin: energy cost $J$
	\item If $i$ is neighbour of $j$, then $j$ is neighbour of $i$, too. But energy cost only once!
	\item 
		\begin{align*}
			E &= -\frac{J}{2V} \sum_x \sum_{y: y \text{ is neighbour of }x} g_x \cdot g_y 
			  = -\frac{J}{2V} \sum_{<x, y>} g_x \cdot g_y
		\end{align*}			
	\item Attention: $x, y$ are \emph{coordinates}; $x$ \enquote{contains} row- \emph{and} column ID
	\end{itemize}
\end{itemize}
%
\end{frame}

% =========================================================================== %

\begin{frame}{Physics: Expectation Value}
%
\begin{itemize}
\item A priori: Spins independent from one another
\item For $V = L \cdot L$ lattice points: $2^V$ possible arrangements
\item Each arrangement: different energy
\item Not all arrangements equally likely!
\item Real world:
	\begin{itemize}
	\item Energy of an object is subject to fluctuations
	\item Thus: any state is possible
	\end{itemize}
\item Probability to realize a particular state depends on its energy! $p = p(E)$
\item Expectation value for magnetization:
	\[ <M> ~ = \sum_{\text{state } G} p[E(G)] \cdot M(G) \]
\end{itemize}
%
\end{frame}

% =========================================================================== %

\begin{frame}{Physics: Boltzmann Distribution}
%
Boltzmann distribution:
\[ p(E) = \frac
	{\exp( -\frac{E}{k_B T} )}
	{\sum_\mathcal{E} \exp( -\frac{\mathcal{E}}{k_B T} )}
\]

Where:
\begin{itemize}
\item $T$: Temperature in Kelvin
\item $k_B$: Boltzmann constant
\end{itemize}

Thus:
\begin{itemize}
\item \emph{In principle} magnetization for eavery temperature can be computed
\item \emph{In practice} too time intensive
\end{itemize}

Outlook:
\begin{itemize}
\item Metropolis algorithm circumvents this problem
\end{itemize}
%
\end{frame}

% =========================================================================== %

\begin{frame}{Programming: From Reality to Code}
%
We need:
\begin{itemize}
\item Given: Temperature
\item A lattice
	\begin{itemize}
	\item Comprising of atoms
	\item Atoms have property \emph{spin}: $+1$ or $-1$
	\item Property: magnetization
	\item Property: total energy
	\item Relation: \enquote{is neighbour of}
	\item Given: initial state
	\end{itemize}
\item Probability to realize a particular state of the lattice
\end{itemize}
%
\end{frame}

% =========================================================================== %

\begin{frame}[fragile]{Programming: Ordering our Analysis}
%
\begin{itemize}
\item Temperature: only important once we get to probabilities
\item Lattice: \enquote{feels like an object} \Thus class
	\begin{itemize}
	\item State $G$: Collection of spins \Thus Array-like (\inPy{list}, NumPy-Array, ...)
		\begin{itemize}
		\item Equal data types ($+1$ or $-1$)
		\item Filled table
		\item[\Thus] NumPy-Array; data type \texttt{np.int64}
		\item Thus, implicitly \emph{Adressing} atoms by their indices
		\end{itemize}
	\item \enquote{Questions to the lattice}: magnetization and energy
		\begin{itemize}
		\item Can be computed from $G$ and $J$ alone
		\item[\Thus] Methods
		\end{itemize}
	\item Neighbour-relation
		\begin{itemize}
		\item Could be computed from coordinates and lattice dimension
		\item But: Needed very frequently
		\item Alternative: Compute once, store in memory
		\item[\Thus] Second NumPy-Array
		\end{itemize}
	\end{itemize}
\end{itemize}
%
\end{frame}

% =========================================================================== %

\begin{frame}{Programming: Select Subproblems}
%
\begin{itemize}
\item We see: our description \enquote{explodes} 
	\begin{itemize}
	\item Already jotted down a lot of notes
	\item Neighbourhood relation not yet fully described
	\item No thoughs at all for coding probabilites
	\end{itemize}
\item But: already quite some concrete thoughs on representation of the lattice
\item[\Thus] Select concrete results, set back the rest
\end{itemize}
%
\end{frame}

% =========================================================================== %

\begin{frame}[fragile]{Programming: class \texttt{Grid} -- first concept}
%
\vspace{-6pt}
\begin{tcolorbox}[title=class \texttt{Grid}]

\vspace{-3pt}
\begin{itemize}
\item Class attributes
	\begin{itemize}
	\item None (If we regard two lattices in the same program, they will have nothing in common)
	\end{itemize}
\item Instance attributes
	\begin{itemize}
	\item NumPy-Array \texttt{spins}
	\item \inPy{int L}: width of the lattice (not indispensible, but \enquote{more comfortable})
	\item \inPy{float J}: coupling constant (necessary for computing the energy)
	\end{itemize}
\item Methods
	\begin{itemize}
	\item \inPy{__init__(self, L, J, ??)} (maybe additional parameters): prepares a lattice in some \emph{initial state}
	\item \texttt{getMagnetization(self)} (no further parameters)
	\item \texttt{getEnergy(self)} (no further parameters)
	\end{itemize}
\end{itemize}
\end{tcolorbox}
%
\end{frame}

% =========================================================================== %

\begin{frame}[fragile]{Programming: class \texttt{Grid} -- first code}
%
\vspace{-6pt}
\begin{codebox}
\begin{minted}[linenos, fontsize=\scriptsize]{python3}
import numpy as np

class Grid :
    def __init__ (self, L, J = 1) :
        self.L     = L
        self.J     = J
        self.spins = np.ones(shape = (L, L), dtype=np.int64)

g = Grid(5)
print(g.spins)
\end{minted}
\end{codebox}
%

\vspace{-12pt}
\begin{hintbox}[Answer open questions as simply as possible]
We haven't covered the \emph{initial state} so far. In such a case: start with a \emph{simple} fill-in, like here: \texttt{np.ones}
\end{hintbox}
%
\end{frame}

% =========================================================================== %

\begin{frame}[fragile]{Retrofit the forgotten part}
%
\begin{itemize}
\item What does \emph{initial state} mean?
	\begin{itemize}
	\item Example: probability for any atom to be spin up
	\item[\Thus] \inPy{float}-argument \texttt{p}: probability
	\end{itemize}
\item Deduce \emph{effects} from this amendment
	\begin{itemize}
	\item Spins now random
	\item[\Thus] no new attributes to the class
	\item[\Thus] \texttt{np.ones} no longer sufficient
	\end{itemize}
\item Include new effect in the method
	\begin{itemize}
	\item Tasks performed by a method should remain simple
	\item \enquote{One thought -- one method}. Rarely more than 20 lines per function/method
	\item New method: pick atoms at random and flip their spin
	\end{itemize}
\end{itemize}
%
\end{frame}

% =========================================================================== %

\begin{frame}[fragile]
%
\begin{codebox}
\begin{minted}[linenos, fontsize=\scriptsize]{python3}
import numpy as np

class Grid :
    def __init__ (self, L, J = 1, p = None) :
        self.L     = L
        self.J     = J
        self.spins = np.ones(shape=(L, L), dtype=np.int64)
        
        if p : self.shake(p)
        
    def shake (self, p = 0.5) :
        mask = np.random.uniform(size=(self.L, self.L)) < p
        self.spins[mask] *= -1

g = Grid(5, 1, 0.5)
print(g.spins)
\end{minted}
\end{codebox}
%
\end{frame}

% =========================================================================== %

\begin{frame}[fragile]{New methods}
%
\begin{tcbraster}[raster columns=2,
                  raster equal height,
                  nobeforeafter,
                  raster column skip=0.5cm]
\begin{codebox}[First planned method]
\begin{minted}[linenos, fontsize=\scriptsize]{python3}
import numpy as np

class Grid :
    # as before
    
    def magnetization (self) :
        return np.sum(self.spins) / \
               self.L ** 2
    
    def energy (self) :
        # ???
        pass

g = Grid(5, 1, 0.5)
print(g.spins)
print(g.magnetization())
\end{minted}
\end{codebox}
%
\begin{cmdbox}[Output]
\begin{minted}[fontsize=\scriptsize]{text}

[[ 1 -1  1 -1 -1]
 [-1  1  1 -1 -1]
 [ 1  1  1 -1  1]
 [ 1  1 -1 -1  1]
 [-1 -1 -1 -1 -1]]
-0.12
\end{minted}
\end{cmdbox}
\end{tcbraster}
%
\end{frame}

% =========================================================================== %

\begin{frame}[fragile]{Neighbour Lists}
%
\begin{itemize}
\item Already discussed: trade memory for processor time
\item Analogous thoughts to construction of the lattice
	\begin{itemize}
	\item List of coordinates
	\item Always same data type, completely filled table
	\item[\Thus] NumPy-Array
	\item \enquote{Table with \inPy{tuple}s as entries} (4 neighbours with row and column)
	\end{itemize}
\item Put it into the code...
	\begin{itemize}
	\item ... as a proper method ...
	\item ... called by \inPy{__init__}
	\end{itemize}
\item Special case: boundaries?
	\begin{itemize}
	\item Different possibilities
	\item Most common, most simple approach: \emph{cyclical boundary conditions}: left boundary connected to the right, upper boundary connected to the bottom one
	\item Alternatively: leave empty, modulus neighbourhood, ...
	\end{itemize}
\end{itemize}
%
\end{frame}

% =========================================================================== %

\begin{frame}[fragile]
%
\vspace{-9pt}
\begin{codebox}
\begin{minted}[linenos, fontsize=\scriptsize]{python3}
class Grid :
    def __init__ (self, L, J = 1, p = None) :
        self.L          = L
        self.J          = J
        self.spins      = np.ones (shape = (L, L)      , dtype = np.int64)
        self.neighbours = np.zeros(shape = (L, L, 4, 2), dtype = np.int64)
        self.makeNeighbourList()
        if p : self.shake(p)
    
    def makeNeighbourList (self) :
        for     row in range(self.L) :
            for col in range(self.L) :
                self.neighbours[row][col][0] = (row - 1, col    )
                self.neighbours[row][col][1] = (row + 1, col    )
                self.neighbours[row][col][2] = (row    , col - 1)
                self.neighbours[row][col][3] = (row    , col + 1)
                
        mask = self.neighbours < 0
        self.neighbours[mask]  += self.L
        mask = self.neighbours >= self.L
        self.neighbours[mask]  -= self.L
\end{minted}
\end{codebox}
%
\end{frame}

% =========================================================================== %

\begin{frame}[fragile]{Optimization}
%
\begin{itemize}
\item Recall: Symmetry of neighbourhod relation
\item Instead of: count each neighbour at half weight: count only half of the neighbours!
\end{itemize}
%
\vspace{-6pt}
\begin{codebox}
\begin{minted}[linenos, fontsize=\scriptsize]{python3}
class Grid :
    def __init__ (self, L, J = 1, p = None) :
        # ...
        self.neighbours = np.zeros(shape=(L, L, 2, 2), dtype=np.int64)
        # ...
    
    def makeNeighbourList (self) :
        for     row in range(self.L) :
            for col in range(self.L) :
                self.neighbours[row][col][0] = (row + 1, col    )
                self.neighbours[row][col][1] = (row    , col + 1)
                
        mask = self.neighbours >= self.L
        self.neighbours[mask]  -= self.L
\end{minted}
\end{codebox}
%
\end{frame}

% =========================================================================== %

\begin{frame}[fragile]
%
\begin{codebox}
\begin{minted}[linenos, fontsize=\scriptsize]{python3}
class Grid :
    # ...
    
    def energy (self) :
        reVal = 0
        
        for     row in range(self.L) :
            for col in range(self.L) :
                neighbours = self.neighbours[row, col]
                for neighbour in neighbours :
                    reVal += self.spins[row, col] * self.spins[tuple(neighbour)]
        
        return reVal / self.L ** 2

g = Grid(5, 1, 0)           # for probability 0 we can predict
print(g.spins)              # the result particularly easy!

print("mag:", g.magnetization())
print(" E :", g.energy())
\end{minted}
\end{codebox}
%
\end{frame}

% =========================================================================== %

\begin{frame}{Algorithmic Effort}
%
\begin{itemize}
\item Now possible in theory:
	\begin{itemize}
	\item Compute lattice energy and associated probability for each configuration
	\item From this, eventually: expectation value
	\end{itemize}
\item Number of possible configurations: $2^V$
	\begin{itemize}
	\item For our \emph{tiny} lattice with 25 atoms: 33\,554\,432 configurations
	\item Assumption: Computation time per configuration \SI{1}{ms}
	\item[\Thus] 9 hours 19 minutes
	\end{itemize}
\item[\Thus]  \emph{Fat Oof}
\end{itemize}
%
\end{frame}

% =========================================================================== %

\begin{frame}{Solution: Monte Carlo Methods}
%
\begin{itemize}
\item Pick a \emph{random} configuration
\item[\Thus] Instead of exploring the entire configuration space: random samples
\item Cf.: Election forecasts
	\begin{itemize}
	\item Sample has to be \emph{representative} of the configuration space
	\item \enquote{Has to follow the correct probability distribution}
	\item Flat distribution: random sampling is sufficient
	\end{itemize}
\item There are \enquote{mathematical tricks} that guarantee a Boltzmann distribution
\item[\Thus] \emph{Metropolis Algorithm}
\end{itemize}
%
\end{frame}

% =========================================================================== %

\begin{frame}{Metropolis Algorithm for Humans}
%
\begin{itemize}
\item Pick a random atom
\item Compute how lattice energy would change if we flip the spin
	\begin{itemize}
	\item If energy goes down: accept in any case
	\item If energy goes up: accept only with probability $p = \exp(-\frac{\Delta E}{k_B T})$
	\end{itemize}
\item Store the resulting configuration (even, if no change was made)
\item Repeat this many times over
\item[\Thus] Get a \enquote{list of configurations} $G_1, G_2, ... G_N$
	\begin{itemize}
	\item Configurations in this list follow the Boltzmann distribution
	\item Average over the measured quantities of these configurations gives expectation value of the quantity
	\item \eg: Get magnetizations $M(G_1), M(G_2), ...$ and compute the average $<M>$. This is an approximation of the \enquote{exact} expectation value
	\end{itemize}
\end{itemize}
%
\end{frame}

% =========================================================================== %

\begin{frame}{Metropolis Algorithm in Python}
%
\begin{center}
(File \texttt{007-Ising-f.py})
\end{center}
%
\end{frame}

% =========================================================================== %

\begin{frame}{Disclaimer and Advertisement}
%
\begin{hintbox}[Lecture \emph{Monte Carlo Methods for Physicists}]
The Code shown before is far from optimal. I didn't want to go too far into specifics since they are covered in more detail in another lecture anyway. The optimized code, however, follows the same train of thought.

\vspace{6pt}
To interested (nature) scientists I can really recommend the lecture \emph{Monte Carlo Methods for Physicists} by Dr. Bloch. There, the Ising Model and the Metropolis algorithm as well as more advanced methods are covered in detail. Part of the lecture are coding exercises in C++.
\end{hintbox}
%
\end{frame}