% =========================================================================== %

\begin{frame}[t,plain]
\titlepage
\end{frame}

% =========================================================================== %

\begin{frame}[fragile]{Our State So Far: All Essential Tools}
%
\begin{itemize}
\item Tools we have so far: enough to write alomost any program
	\begin{itemize}
	\item We only lack the capability to access \enquote{devices} such as the screen or hard disks
	\item Two commands (that nobody uses any longer)
	\item Enough to write entire operating systems
	\end{itemize}
\item Art of Programming: How to combine these tools
	\begin{itemize}
	\item Form of art, takes time to devellop
	\item Aim here: Learn from examples
	\end{itemize}
\item Train of Thought
	\begin{itemize}
	\item Analysis: decompose a problem into its atoms
	\item Synthesis: solve the atomic problems, combine them into a complete program
	\end{itemize}
\end{itemize}
%
\end{frame}

% =========================================================================== %

\begin{frame}[fragile]{Mini Project: Black Jack}
%
\begin{itemize}
\item Recall Rules: Black Jack (simplified)
\item Play with six decks of cards
	\begin{itemize}
	\item 6 $\times$ 52 cards, 4 suits, values A, 2..10, J, Q, K
	\end{itemize}
\item Players make a bet, put money into the pot
\item They draw cards in turn
	\begin{itemize}
	\item After each card they can decide to keep drawing or stop
	\item If the sum of the values is greater than 21, a player looses
	\item If everyone has stopped or lost, the player(s) closest to 21 win(s)
	\end{itemize}
\item Today: Only up to representing cards and players
\item Tomorrow: Game mechanics
\end{itemize}
%
\end{frame}

% =========================================================================== %

\begin{frame}[fragile]{Analysis}
%
What objects and concepts do we have to represent in code?
\begin{itemize}
\item Cards
\item Money
\item Players or their actions
\item Win or loss condition
\end{itemize}
%
\end{frame}

% =========================================================================== %

\begin{frame}[fragile]{Analysis}
%
How do we portray them?
\begin{itemize}
\item Money: simple number \thus \inPy{int} or \inPy{float} variable
\item Cards
	\begin{itemize}
	\item Collection of alike objects
	\item One Card: suit and value
	\item Limited Range
	\item Two Strings representing suit and value, respectively
	\item Suit and Value-Number belong together \thus pack them as a unit \thus \inPy{tuple}
	\item \emph{Collection} \thus \inPy{list}
	\end{itemize}
\end{itemize}
%
\begin{hintbox}[Container Types]
If in doubt which container is the correct one, use a \inPy{list}.

The choice of a \inPy{tuple} for a card is more convention than necessity; a \inPy{list} would do just as well.
\end{hintbox}
%
\end{frame}

% =========================================================================== %

\begin{frame}[fragile]
%
%
\begin{columns}[T]
\column{.5\linewidth}
\begin{Large}
	{Analysis}
\end{Large}

How do we portray players?
\begin{itemize}
\item Several of them \Thus \inPy{list}
\item Players have a name
	\begin{itemize}
	\item[\Thus] String variable for each Player
	\end{itemize}
\item Players have money
	\begin{itemize}
	\item[\Thus] Money-Variable for each Player
	\end{itemize}
\item Players have Cards
	\begin{itemize}
	\item[\Thus] Cards-List for each player
	\end{itemize}
\item Players need to decide on actions
	\begin{itemize}
	\item[\Thus] Something with \inPy{input}
	\item[\Thus] Own function to keep things tight
	\end{itemize}
\item Players draw cards
	\begin{itemize}
	\item[\Thus] Function to transfer cards from the deck
	\end{itemize}
\end{itemize}
%
\column{.5\linewidth}
\begin{hintbox}[Non-Object Concepts]
Our analysis shows: players are rather complex things, as they are capable of performing \emph{actions}.

It is best to start coding from passive objects -- mere collections of information. Everything else we set back for a second.

Here, we identify the object-characteristics of the player: their name, money, cards. Everything else will be modelled thereafter, as actions will need to access these information.
\end{hintbox}
\end{columns}
%
\end{frame}
	
% =========================================================================== %

\begin{frame}[fragile]{Analysis}
%
How do we portray players?
%
\begin{itemize}
\item We found: three kinds of values (name, money, cards)
\item[\Thus] Some Container
\item Possible: \inPy{list}, but...
\item No natural ordering. Should \inPy{player[0]} be name, money, cards?
\item[\Thus] \inPy{dict}! Allows \inPy{player['money']}
\end{itemize}
%
\end{frame}

% =========================================================================== %

\begin{frame}[fragile]{Synthesis (Attempt)}
%
\begin{codebox}[First Few Lines of Code (Non-Functional)]
\begin{minted}[linenos, fontsize=\scriptsize]{python}
deck = [(...), (...), ...]
players = [
    {'name' : 'Jacques Hartington', 'money' : 9001, 'cards' : [(...), (...), ...]},
    {'name' : 'Farin Urlaub'      , 'money' : 9001, 'cards' : [(...), (...), ...]},
    ...
]
\end{minted}
\end{codebox}
%
\begin{itemize}
\item Find new problem: How do we get a deck?
\item Back into analysis mode
\item Leave lists empty for now (simple code, no error message)
\end{itemize}
%
\end{frame}

% =========================================================================== %

\begin{frame}[fragile]{Analysis}
%
How do we build a deck?
%
\begin{itemize}
\item We know which cards there are
	\begin{itemize}
	\item We could list them manually \Thus a lot of typing, but possible
	\item They have a regular structure \Thus loops
		\begin{itemize}
		\item Same 13 cards for all four suits
		\item One for each value between 1 and 13
		\end{itemize}
	\end{itemize}
\item They should be shuffled
	\begin{itemize}
	\item Have a look at the \texttt{random} module
	\end{itemize}
\end{itemize}
%
\end{frame}

% =========================================================================== %

\begin{frame}[fragile]{Synthesis: Some Test Code}
%
\begin{codebox}[Attempt to Build a Deck (1)]
\begin{minted}[linenos, fontsize=\scriptsize]{python}
deck = []
for deckID in range(6) :
  for suit in ("Hearts", "Diamond", "Club", "Spades") :
      for value in ("A", "2", "3", "4", "5", "6", "7", "8", "9", "10", "J", "Q", "K") :
          deck.append( (suit, value) )

print(deck)
\end{minted}
\end{codebox}
%
\begin{cmdbox}[Output: Attempt to Build a Deck (1)]
\begin{minted}[fontsize=\scriptsize]{text}
[('Hearts', 'A'), ('Hearts', '2'), ('Hearts', '3'), ('Hearts', '4'), ('Hearts', '5'),
 ('Hearts', '6'), ('Hearts', '7'), ('Hearts', '8'), ('Hearts', '9'), ('Hearts', '10'),
 ('Hearts', 'J'), ('Hearts', 'Q'), ('Hearts', 'K'), ('Diamond', 'A'), ('Diamond', '2'),
 ...
\end{minted}
\end{cmdbox}
%
\end{frame}

% =========================================================================== %

\begin{frame}[fragile]{Code Review: Making Things Look Nicer}
%
\begin{itemize}
\item \inPy{for} loop saved us typing 312 card definitions by hand
\item Still: 13 values explicitly stated
\item We don't make use of the regularity of the values!
\item Numbers: Value correspons with String directly!
\item For Royals: need lookup!
\item[\Thus] \inPy{str} and \inPy{dict} and a function!
\end{itemize}
%
\end{frame}

% =========================================================================== %

\begin{frame}[fragile]{Synthesis: Improved Deck Builder}
%
\begin{codebox}[Attempt to Build a Deck (2)]
\begin{minted}[linenos, fontsize=\scriptsize]{python}
deck = []

def lookupCardName (value) :
    deckTexts = {1: "A", 11 : "J", 12 : "Q", 13 : "K"}
    
    if value in deckTexts :
        return deckTexts[value]
    else :
        return str(value)

for deckID in range(6) :
  for suit in ("Hearts", "Diamond", "Club", "Spades") :
      for value in ( range(1, 14) ) :
          deck.append( (suit, lookupCardName(value)) )

print(deck)
\end{minted}
\end{codebox}
%
\end{frame}

% =========================================================================== %

\begin{frame}[fragile]{Synthesis: Shuffling the Deck}
%
\begin{cmdbox}[Finding Functions]
\begin{minted}[fontsize=\scriptsize]{text}
>>> dir(random)
['BPF', 'LOG4', 'NV_MAGICCONST', 'RECIP_BPF', 'Random', 'SG_MAGICCONST', 'SystemRandom',
 'TWOPI', '_Sequence', '_Set', '__all__', '__builtins__', '__cached__', '__doc__', 
 '__file__', '__loader__', '__name__', '__package__', '__spec__', '_accumulate', '_acos',
 '_bisect', '_ceil', '_cos', '_e', '_exp', '_inst', '_log', '_os', '_pi', '_random',
 '_repeat', '_sha512', '_sin', '_sqrt', '_test', '_test_generator', '_urandom', '_warn',
 'betavariate', 'choice', 'choices', 'expovariate', 'gammavariate', 'gauss', 'getrandbits',
 'getstate', 'lognormvariate', 'normalvariate', 'paretovariate', 'randint', 'random',
 'randrange', 'sample', 'seed', 'setstate', 'shuffle', 'triangular', 'uniform',
 'vonmisesvariate', 'weibullvariate']
\end{minted}
\end{cmdbox}
%
... I wonder what {\color{blue}\texttt{shuffle}} does...
%
\end{frame}

% =========================================================================== %

\begin{frame}[fragile]{Synthesis: Shuffling the Deck}
%
\begin{cmdbox}[Learning About Functions]
\begin{minted}[fontsize=\scriptsize]{text}
>>> help(random.shuffle)
Help on method shuffle in module random:

shuffle(x, random=None) method of random.Random instance
    Shuffle list x in place, and return None.
    
    Optional argument random is a 0-argument function returning a
    random float in [0.0, 1.0); if it is the default None, the
    standard random.random will be used.

\end{minted}
\end{cmdbox}
%
... guess what? It shuffles a list! Exactly what we need!
%
\end{frame}

% =========================================================================== %

\begin{frame}[fragile]
%
\begin{codebox}[Attempt To Build a Deck (3)]
\begin{minted}[linenos, fontsize=\scriptsize]{python}
import random

deck = []
def lookupCardName (value) :
    # as before
for deckID in range(6) :
    # as before

random.shuffle(deck)    # look how nice this reads! Like "normal" English!

print(deck)
\end{minted}
\end{codebox}
%
\begin{cmdbox}[Output: Attempt To Build a Deck (3)]
\begin{minted}[fontsize=\scriptsize]{text}
[('Hearts', 'K'), ('Hearts', '5'), ('Spades', '10'), ('Spades', 'K'), ('Hearts', '6'),
 ('Club', '5'), ('Spades', '5'), ('Diamond', '1'), ('Diamond', '4'), ('Club', '3'), 
 ('Diamond', '3'), ('Club', '4'), ('Hearts', '9'), ('Spades', '8'), ('Diamond', '8'), 
 ('Hearts', '1'), ('Spades', '6'), ('Hearts', '8'), ('Spades', 'Q'), ('Spades', '1'), 
 ...
\end{minted}
\end{cmdbox}
%
\end{frame}

% =========================================================================== %

\begin{frame}[fragile]{Analysis: Dealing Cards to the Players}
%
\begin{itemize}
\item Real life game: Croupier deals cards to the players
	\begin{itemize}
	\item Remove top card from deck
	\item Add it to one player's hand
	\end{itemize}
\item So we need in code...
	\begin{itemize}
	\item Way to access the deck -- \inPy{deck[index]}
	\item Way to Remove a card
		\begin{itemize}
		\item One approach: \inPy{deck = deck[from_index : to_index]}
		\item Better: \inPy{card = deck.pop(index)} (combines accessing and removing)
		\item \texttt{pop}: argument \texttt{index} is optional. If omitted, removes/returns last element in the list.
		\end{itemize}
	\end{itemize}
\item Way to put the card into a player's hands
	\begin{itemize}
	\item \inPy{list_of_some_players_cards.append(card)}
	\end{itemize}
\end{itemize}
%
\end{frame}

% =========================================================================== %

\begin{frame}[fragile]{Synthesis: Dealing Cards to the Players}
%
\begin{itemize}
\item We need access to:
	\begin{itemize}
	\item The deck -- variable \texttt{deck}
	\item One player's hand
		\begin{itemize}
		\item First: Get one player: \inPy{players[ID]}
		\item Then: Select their cards: \inPy{players[ID]['cards']}
		\end{itemize}
	\end{itemize}
\item We perform the action
	\begin{itemize}
	\item \inPy{players[ID]['cards'].append( deck.pop() )}
	\end{itemize}
\item Perform the action \emph{for each player}
	\begin{itemize}
	\item \inPy{for} loop!
	\item Indices: possible\\
		\inPy{for ID in range(len(players))}
	\item Better: References\\
		\inPy{for player in players :}\\
		~~~~\inPy{player['cards'].append( deck.pop() )} 
	\end{itemize}
\end{itemize}
%
\end{frame}

% =========================================================================== %

\begin{frame}[fragile]
%
\begin{codebox}[Dealing Cards to the Players]
\begin{minted}[linenos, fontsize=\scriptsize]{python}
deck = ... #as before

players = [
    {'name' : 'Jacques Hartington', 'money' : 9001, 'cards' : []},
    {'name' : 'Farin Urlaub'      , 'money' : 9001, 'cards' : []},
 ]

for player in players :
    player['cards'].append( deck.pop() )

print(players)
\end{minted}
\end{codebox}
%
\begin{cmdbox}[Output: Dealing Cards to the Players]
\begin{minted}[fontsize=\scriptsize]{text}
[{'name': 'Jacques Hartington', 'money': 9001, 'cards': [('Hearts', '9')]},
 {'name': 'Farin Urlaub', 'money': 9001, 'cards': [('Diamond', '6')]}]
\end{minted}
\end{cmdbox}
%
\end{frame}

% =========================================================================== %

\begin{frame}{Code Design: One Function Per Idea}
%
\begin{itemize}
\item High level of abstraction
\item Real life objects into lists and numbers
\item Re-Introduce human language
\item Group Ideas into Functions
\end{itemize}
%
\end{frame}

% =========================================================================== %

\begin{frame}[fragile]
%
\begin{codebox}[Code in Functions ...]
\begin{minted}[linenos, fontsize=\scriptsize]{python}
import random

# =========================================================================== #

def lookupCardName (value) :
    deckTexts = {11 : "J", 12 : "Q", 13 : "K"}
        
    if value in deckTexts : return deckTexts[value]
    else                  : return str(value)
        
def buildDeck() :
    deck = []

    for deckID in range(6) :
        for suit in ("Hearts", "Diamond", "Club", "Spades") :
            for value in ( range(1, 14) ) :
                deck.append( (suit, lookupCardName(value)) )
    
    random.shuffle(deck)    # look how nice this reads! Like "normal" English!
    return deck
\end{minted}
\end{codebox}
%
\end{frame}

% =========================================================================== %

\begin{frame}[fragile]
%
\begin{codebox}[Code in Functions (Continued)]
\begin{minted}[linenos, firstnumber=last, fontsize=\scriptsize]{python}

def dealCards (players, deck) :
    for player in players :
        player['cards'].append( deck.pop() )

# =========================================================================== #

deck = buildDeck()

players = [
    {'name' : 'Jacques Hartington', 'money' : 9001, 'cards' : []},
    {'name' : 'Farin Urlaub'      , 'money' : 9001, 'cards' : []},
 ]

dealCards(players, deck)

print(players)
\end{minted}
\end{codebox}
%
\end{frame}

% =========================================================================== %

\begin{frame}{Getting Rid of Last Hard Coded Elements}
%
\begin{itemize}
\item Current Form: Always players Jacques Hartington and Farin Urlaub
\item Desired: Flexible Code
\item[\Thus] User Input: Who plays?
\item Keep Function Form: new idea \enquote{add player}
\end{itemize}
%
\end{frame}

% =========================================================================== %

\begin{frame}{Analysis: Adding a New Player}
%
\begin{itemize}
\item Need to know:
	\begin{itemize}
	\item Names and Money of players
	\item[\Thus] \inPy{input}
	\end{itemize}
\item Need \emph{not} to know:
	\begin{itemize}
	\item Cards -- they all start with an empty hand!
	\end{itemize}
\item There's more than one player
	\begin{itemize}
	\item[\Thus] Some loop
	\item Either: ask for number of players first \Thus \inPy{for ID in range(playerCount)}
	\item Or: empty line signifies end of list
	\end{itemize}
\end{itemize}
%
\end{frame}

% =========================================================================== %

\begin{frame}[fragile]
%
\begin{codebox}[Adding a new player]
\begin{minted}[linenos, fontsize=\scriptsize]{python}
def registerPlayers() :
    players = []
    notDoneYet = True
    
    while notDoneYet :
        name = input("Please enter the next player's name "
                     "(empty line ends player registration): ")
        
        if len(name) == 0 :   # or: if name == "", or if not len(name), or ...
            notDoneYet = False
            break
        
        money = float( input(f"Please enter {name}'s start money: ") )
        
        players.append(
            {'name' : name, 'money' : money, 'cards' : []}
        )
    
    return players
\end{minted}
\end{codebox}
%
\end{frame}

% =========================================================================== %

\begin{frame}[fragile]
%
\begin{codebox}[Adding a New Player -- Adjustments to Module Level Code]
\begin{minted}[linenos, fontsize=\scriptsize]{python}
deck = buildDeck()
players = registerPlayers()

dealCards(players, deck)

print(players)
\end{minted}
\end{codebox}
%
\begin{cmdbox}[Output: Adding a New Player -- Adjustments to Module Level Code]
\begin{minted}[fontsize=\scriptsize]{text}
Please enter the next player's name (empty line ends player registration): Dusky Joe
Please enter Dusky Joe's start money: 9001
Please enter the next player's name (empty line ends player registration): Blue Chameleon
Please enter The Chameleon's start money: 9001
Please enter the next player's name (empty line ends player registration):
[{'name': 'Dusky Joe', 'money': 9001.0, 'cards': [('Spades', '4')]},
 {'name': 'Blue Chameleon', 'money': 9001.0, 'cards': [('Club', '10')]}]
\end{minted}
\end{cmdbox}
%
\end{frame}